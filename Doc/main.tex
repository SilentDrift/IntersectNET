\documentclass[conference]{IEEEtran}

% Packages
\usepackage{amsmath,amsfonts,amssymb}
\usepackage{graphicx}
\usepackage{cite}
\usepackage{hyperref}
\usepackage{balance} % For balancing the last page columns

% Title and Author Information
\title{IntersectNET: Traffic Flow? Managed}

\author{
    \IEEEauthorblockN{Arshia Gharooni}
    \IEEEauthorblockA{
        Email: arshiyagharoony@gmail.com
    }
}

\begin{document}

\maketitle

\begin{abstract}
Efficient traffic management at intersections is a critical factor in reducing congestion and improving urban mobility. This paper introduces \textit{IntersectNET}, a comprehensive traffic simulation and optimization framework for managing intersections with four streets, each with two lanes (one inbound and one outbound). We model lane dynamics using macroscopic traffic flow equations and optimize signal timings using gradient descent. The primary objective is to minimize the \textit{total travel time} (TTT) while maintaining realistic lane-changing behaviors and managing intersection flow. This paper presents the mathematical formulation, simulation results, and optimization methodology in detail.
\end{abstract}

\IEEEpeerreviewmaketitle

\section{Introduction}

Traffic congestion in urban areas has long been a problem that requires optimized management of intersections. Efficient intersection control, especially in scenarios with multiple lanes and traffic directions, can significantly reduce travel delays and vehicle congestion. This paper focuses on simulating and optimizing a four-street intersection using the \textit{IntersectNET} framework, where each street has two lanes: one for incoming traffic and one for outgoing traffic. Our goal is to dynamically control signal timings at the intersection while modeling realistic lane-changing behavior.

\section{Mathematical Modeling}

\subsection{Lane Dynamics}

Lanes are modeled using macroscopic traffic flow equations, where traffic is treated as a continuous fluid. The \textit{Second-Order Macroscopic Traffic Flow Model} is used, which accounts for both vehicle density and velocity dynamics.

\subsubsection{Continuity Equation (Conservation of Mass)}
The continuity equation ensures that the number of vehicles within a lane remains conserved over time:

\begin{equation}
\frac{\partial \rho}{\partial t} + \frac{\partial q}{\partial x} = 0
\end{equation}

Where:
\begin{itemize}
    \item $\rho(x, t)$: Vehicle density (vehicles per meter)
    \item $q(x, t) = \rho v$: Traffic flow (vehicles per second)
    \item $v(x, t)$: Average vehicle speed (meters per second)
\end{itemize}

This equation ensures that changes in density are balanced by changes in vehicle flow.

\subsubsection{Momentum Equation (Conservation of Momentum)}
The momentum equation adjusts vehicle velocities based on traffic pressure and relaxation towards equilibrium:

\begin{equation}
\frac{\partial v}{\partial t} + v \frac{\partial v}{\partial x} = -\frac{1}{\rho} \frac{\partial p}{\partial x} + \frac{V_{\text{eq}}(\rho) - v}{\tau}
\end{equation}

Where:
\begin{itemize}
    \item $p$: Traffic pressure, modeled as $p = k \rho^\gamma$ with constants $k$ and $\gamma$
    \item $V_{\text{eq}}(\rho)$: Equilibrium speed, dependent on vehicle density
    \item $\tau$: Relaxation time
\end{itemize}

\subsubsection{Equilibrium Speed and Relaxation}
The equilibrium speed $V_{\text{eq}}(\rho)$ models how vehicles adjust their speed based on density:

\begin{equation}
V_{\text{eq}}(\rho) = v_{\text{free}} \left( 1 - \frac{\rho}{\rho_{\text{max}}} \right)
\end{equation}

Where:
\begin{itemize}
    \item $v_{\text{free}}$: Free-flow speed (vehicles move without obstruction)
    \item $\rho_{\text{max}}$: Maximum jam density
\end{itemize}

\subsection{Intersection Dynamics}

At the intersection, transition matrices govern how incoming traffic is distributed across outgoing lanes.

\subsubsection{Transition Matrix}
Each movement at the intersection (straight, left, right) is modeled using a transition matrix $T$. For an intersection with four streets (each with an inbound and outbound lane), the transition matrix is structured as follows:

\begin{equation}
T_{ij} = \begin{pmatrix}
T_{11} & T_{12} & T_{13} & T_{14} \\
T_{21} & T_{22} & T_{23} & T_{24} \\
T_{31} & T_{32} & T_{33} & T_{34} \\
T_{41} & T_{42} & T_{43} & T_{44}
\end{pmatrix}
\end{equation}

Where $T_{ij}$ represents the fraction of flow from incoming lane $i$ directed towards outgoing lane $j$. The sum of each column equals 1 to ensure all incoming flow is distributed.

\subsubsection{Movement Modeling}
Each movement at the intersection is represented by flow dynamics based on the transition matrix. For each incoming lane, there are three possible movements: straight, left turn, and right turn.

\subsubsection{Signal Phasing and Timing}
Signal timings are grouped into non-conflicting phases:
\begin{itemize}
    \item \textbf{Phase 1}: Straight and right turns for streets 1 and 3.
    \item \textbf{Phase 2}: Straight and right turns for streets 2 and 4.
    \item \textbf{Phase 3}: Left turns for streets 1 and 3.
    \item \textbf{Phase 4}: Left turns for streets 2 and 4.
\end{itemize}
Each signal phase is assigned a green time $G_p$, which is dynamically optimized to minimize total travel time.

\subsection{Dynamic Lane Changing}
Lane-changing rates $\sigma_{ij}(t)$ between adjacent lanes are based on differences in speed and density:

\begin{equation}
\sigma_{ij}(t) = \beta \left[ \frac{v_j(t) - v_i(t)}{v_{\text{max}}} + \frac{\rho_i(t) - \rho_j(t)}{\rho_{\text{max}}} \right]^+
\end{equation}

Where $\beta$ is a sensitivity parameter.

\section{Optimization Methodology}

\subsection{Objective Function}
The objective is to minimize the \textit{total travel time} (TTT) across all lanes:

\begin{equation}
\text{TTT} = \sum_{\text{lanes}} \sum_{t=1}^{T} \rho(t) \cdot \Delta x
\end{equation}

Where $\rho(t)$ is the density at time $t$ and $\Delta x$ is the lane length.

\subsection{Gradient Descent Optimization}
The green times for each signal phase are updated using gradient descent. The gradient of the TTT with respect to the green times $G_p$ is given by:

\begin{equation}
\frac{\partial \text{TTT}}{\partial G_p} = \frac{\text{TTT}(G_p + \Delta G) - \text{TTT}(G_p)}{\Delta G}
\end{equation}

Green times are updated iteratively:

\begin{equation}
G_p^{(n+1)} = G_p^{(n)} - \alpha \cdot \frac{\partial \text{TTT}}{\partial G_p}
\end{equation}

\section{Simulation Setup}

\subsection{Network Configuration}
\begin{itemize}
    \item \textbf{4 Streets}, each with 2 lanes (1 inbound, 1 outbound).
    \item \textbf{Intersection} at the center, with straight, left turn, and right turn movements.
\end{itemize}

\subsection{Initialization}
\begin{itemize}
    \item \textbf{Initial Density}: Randomized between 0.01 and 0.02 vehicles per meter.
    \item \textbf{Initial Speed}: Set to free-flow speed of 25 m/s.
\end{itemize}

\section{Results}

\subsection{Density and Speed Heatmaps}
The density heatmap shows how traffic congestion evolves across lanes over time. Speed heatmaps highlight areas where congestion leads to reduced vehicle speeds.

\subsection{Flow Over Time}
The flow of vehicles through each lane is plotted to show how traffic volumes change dynamically over the simulation period.

\subsection{Queue Lengths at Intersection}
Queue lengths at each inbound lane are monitored over time, revealing bottlenecks and the impact of signal timings on intersection performance.

\subsection{Total Travel Time}
The optimization successfully reduces the total travel time (TTT), improving traffic flow efficiency.

\section{Conclusion}
This paper presents \textit{IntersectNET}, a traffic flow management framework for optimizing signal timings at intersections. By using macroscopic traffic flow models and gradient descent optimization, we significantly reduced the total travel time at a four-street intersection. Future work could expand the model to multiple intersections or incorporate real-time traffic data for dynamic control.

\balance

\end{document}
